\chapter{Programs}
\label{ch:programs}

In the following we give a brief description of the main programs used to 
implement the GNSS analysis used in the platform. Note that all of the programs 
depend on the \nameref{ch:pybern} module, hence \ul{make sure that you have 
the lastest version of pybern installed} (see \nameref{sec:pybern-installation}).

A complete list of the available programs can be found in the \verb|bin| folder. 
The following remarks apply fro the programs:
\begin{itemize}
    \item all of the programs are standalone
    \item all of the programs include \href{https://en.wikipedia.org/wiki/Shebang_(Unix)}{shebang}, 
    hence you should be able to run them directly ommiting the interpreter (aka 
    \verb|syncwbern52.py| is equivalent to \verb|python syncwbern52.py|)
    \item all of the programs include a help message, triggered by \verb|-h| or 
    \verb|--help|
    \item all of the programs can be run in \emph{verbose} mode, printing debug messages, via 
    the \verb|--verbose| switch
    \item for Python programs, no version is enforced, the shebang call the default 
    interpreter/version. Should you want to run a program with a different version/interpreter 
    you are free to do so
\end{itemize}

\section{syncwbern52}
\label{sec:programs-syncwbern52}
\verb|syncwbern52| syncronizes (aka mirrors) a local folder, which should normaly 
be the local \verb|/GEN|\index{GEN} folder, to the (remote) AIUB's \verb|/GEN| folder 
located at \url{ftp.aiub.unibe.ch/BSWUSER52/GEN}. This process 
\ul{excludes all *.EPH files} which are system-dependent, binary file(s).
You can see the help message for more information.

\section{get\_vmf1\_grd}
\label{sec:programs-get-vmf1-grd}
\verb|get_vmf1_grd| downloads VMF1\index{VMF1} grid files to be used in the GNSS analysis 
for troposphere estimation/mitigation. Grid files are downloaded from 
\url{VMF1 grid files from https://vmf.geo.tuwien.ac.at/trop_products/GRID/2.5x2/VMF1}.
The script allows for downloading both final and forecast grid files, but note that 
for the latter you will need \nameref{ch:credentials}. You can see the 
help message for more information.


\subsection{Examples}
\label{ssec:programs-get-vmf1-grd-examples}
Download final VMF1\index{VMF1} grid files, for the date 01/01/2021, merge them (all four) to 
a file named \verb|VMF_01012021.GRD| and delete the individual hourly files.\\
\verb|$>get_vmf1_grd.py -y 2021 -d 1 -m VMF_01012021.GRD --del-after-merge|\\
\\

Download (forecast) VMF1\index{VMF1} grid files for today, merge them (all four) to a file 
named \verb|VMF_today.GRD| and delete the individual hourly files. This will need 
credentials to access the forecast VMF1 files, see \nameref{ch:credentials}.\\
\verb|$>date +"%Y-%j" ## get year and doy in Unix-like systems|\\
\verb|2021-268|\\
\verb|## call the program with today's date|\\
\verb|$>get_vmf1_grd.py -y 2021 -d 268 -f -c dso_credentials -m VMF_today.GRD --del-after-merge|\\

\section{getdcb}
\label{sec:programs-getdcb}
\verb|getdcb| is a program that allows downloading of GPS/GNSS Code Differential 
Bias (aka DCB\index{DCB}) files from derived from AIUB. These files can contain 
numerous combinations of biases related to different satellite systems, code observables, 
day of estimation, etc. An exhustive description is thus not possible. The program 
can download any of them, given the correct combination of command line parameters 
by the user.

To check the available DCB files that can be downloaded, you can pass in the \verb|-l| or 
\verb|--list-products| switch and the program will print out a list of the available 
DCBs and how to target them (it will download nothing though).

\subsection{Examples}
\label{ssec:programs-getdcb}
Let's say we want to download the file \verb|CODE_FULL.DCB|, which contains biases 
for both GPS and GLONASS between code obervables P1P2, P1C1 and P2C2. The easiest way, 
is to first list our options:
\verb|$>getdcb.py -l|\\
This will print out a list of files and directions; we target the ones of interest, aka:
\begin{verbatim}
  _Available files in FTP____________________________________________________
  [...]
  CODE_FULL.DCB     Combination of P1P2.DCB, P1C1.DCB (GPS satellites),
                    P1C1_RINEX.DCB (GLONASS satellites), and P2C2_RINEX.DCB
  [...]
  _Arguments for Products____________________________________________________
  [9] type=current, span=monthly, obs=full    | CODE_FULL.DCB (merged [2], [3], [6] and [7])
\end{verbatim}
So, we can now run:\\
\verb|$>getdcb.py -y 2021 -d 268 --type current --time-span monthly --code-type full|\\
\\

Let's say we want the combined GPS and GLONASS monthly P1-P2 DCB values for a past 
date, aka 01/01/2021, that is the first month of 2021. We list again (\verb|$>getdcb.py -l|) 
and see that we have to run (for the \verb|P1P2yymm_ALL.DCB.Z| file):\\
\verb|getdcb.py -y 2021 -d 1 --type final --time-span monthly --code-type p1p2all|\\

\emph{Note that in both above examples we could have skipped the} \verb|--time-span monthly| \emph{option, 
since this is the default (you can see this in the help message). We only add it here for completeness.}


\section{geterp}
\label{sec:programs-geterp}
\verb|geterp| is a program that allows downloading of Earth Rotation Parameter information 
files (aka ERP\index{ERP}), published by AIUB. Users can choose between a variety of 
such files (depending e.g. on the product type --final, rapid, etc-- and the time-span 
they cover). The program  can download any of them, given the correct combination of 
command line parameters by the user.

To check the available ERP files that can be downloaded, you can pass in the \verb|-l| or 
\verb|--list-products| switch and the program will print out a list of the available 
ERPs and how to target them (it will download nothing though).

\subsection{Examples}
\label{ssec:programs-geterp}
Let's say we want to download weekly ERP for a past date, namely 01/01/2021; we 
can run the command:\\
\verb|geterp.py -y 2021 -d 1 --time-span weekly -t final|\\
which will download the file \verb|COD21387.ERP.Z|. If instead we want the respective 
(final) file but with daily (not weekly) records, we can run:\\
\verb|geterp.py -y 2021 -d 1 --time-span daily -t final|\\
which will download the file \verb|COD21385.ERP.Z|.
\\

If the user want to download a file for a date that is too close to current (which 
means that probably the final solution is not yet available), then we can choose a 
number of solution types; the program will try each solution type and if the respective 
file is available it will be downloaded. E.g.\\
\verb|geterp.py -y 2021 -d 266 -t final,final-rapid,early-rapid,ultra-rapid,prediction|\\
which will try (in turn) to download files for \emph{final, final-rapid, early-rapid, ultra-rapid}  
and finally \emph{prediction} solutions. Obviously, users can target each of these 
types individualy, using the target solution as option to the \verb|type| switch.

\section{getion}
\label{sec:programs-getion}
\verb|getion| is a program that allows downloading of ionospheric information 
files published by AIUB. Ionospheric information files are published in one of two 
formats, namely IONEX\index{IONEX} format (normally with extension \verb|INX| or 
\verb|YYI|, with YY the two-digit year) or an internal Bernese format (aka 
\verb|ION|\index{ION}). Users can choose the format they prefer via the \verb|--format| 
switch. A variety of ionospheric products is made available by AIUB, and the ones 
available for downloading can be listed using the \verb|-l| switch.

\subsection{Examples}
\label{ssec:programs-getion}
Let's say we want to download ionospheric information in Bernese format for a past date, 
namely 01/01/2021; we can run the command:\\
\verb|getion.py -y 2021 -d 1 --format bernese --type final|\\
which will download the file \verb|COD21385.ION.Z|.
\\

If the user want to download a file for a date that is too close to current (which 
means that probably the final solution is not yet available), then we can choose a 
number of solution types; the program will try each solution type and if the respective 
file is available it will be downloaded. E.g.\\
\verb|getion.py -y 2021 -d 266 -f bernese -t final,rapid,ultra-rapid,prediction|\\
which will try (in turn) to download files for \emph{final, rapid, ultra-rapid} and 
finally \emph{prediction} solutions. Obviously, users can target each of these types individualy, 
using the target solution as option to the \verb|type| switch.